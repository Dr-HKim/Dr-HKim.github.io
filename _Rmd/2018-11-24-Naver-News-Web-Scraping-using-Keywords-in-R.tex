\documentclass[]{article}
\usepackage{lmodern}
\usepackage{amssymb,amsmath}
\usepackage{ifxetex,ifluatex}
\usepackage{fixltx2e} % provides \textsubscript
\ifnum 0\ifxetex 1\fi\ifluatex 1\fi=0 % if pdftex
  \usepackage[T1]{fontenc}
  \usepackage[utf8]{inputenc}
\else % if luatex or xelatex
  \ifxetex
    \usepackage{mathspec}
  \else
    \usepackage{fontspec}
  \fi
  \defaultfontfeatures{Ligatures=TeX,Scale=MatchLowercase}
\fi
% use upquote if available, for straight quotes in verbatim environments
\IfFileExists{upquote.sty}{\usepackage{upquote}}{}
% use microtype if available
\IfFileExists{microtype.sty}{%
\usepackage{microtype}
\UseMicrotypeSet[protrusion]{basicmath} % disable protrusion for tt fonts
}{}
\usepackage[margin=1in]{geometry}
\usepackage{hyperref}
\hypersetup{unicode=true,
            pdftitle={R 뿉꽌 썝븯뒗 궎썙뱶쓽 꽕씠踰\textless{}84\textgreater{} 돱뒪瑜 쎒겕濡ㅻ쭅(뒪겕옒븨) 븯뒗 諛⑸쾿},
            pdfauthor={hkim},
            pdfborder={0 0 0},
            breaklinks=true}
\urlstyle{same}  % don't use monospace font for urls
\usepackage{color}
\usepackage{fancyvrb}
\newcommand{\VerbBar}{|}
\newcommand{\VERB}{\Verb[commandchars=\\\{\}]}
\DefineVerbatimEnvironment{Highlighting}{Verbatim}{commandchars=\\\{\}}
% Add ',fontsize=\small' for more characters per line
\usepackage{framed}
\definecolor{shadecolor}{RGB}{248,248,248}
\newenvironment{Shaded}{\begin{snugshade}}{\end{snugshade}}
\newcommand{\AlertTok}[1]{\textcolor[rgb]{0.94,0.16,0.16}{#1}}
\newcommand{\AnnotationTok}[1]{\textcolor[rgb]{0.56,0.35,0.01}{\textbf{\textit{#1}}}}
\newcommand{\AttributeTok}[1]{\textcolor[rgb]{0.77,0.63,0.00}{#1}}
\newcommand{\BaseNTok}[1]{\textcolor[rgb]{0.00,0.00,0.81}{#1}}
\newcommand{\BuiltInTok}[1]{#1}
\newcommand{\CharTok}[1]{\textcolor[rgb]{0.31,0.60,0.02}{#1}}
\newcommand{\CommentTok}[1]{\textcolor[rgb]{0.56,0.35,0.01}{\textit{#1}}}
\newcommand{\CommentVarTok}[1]{\textcolor[rgb]{0.56,0.35,0.01}{\textbf{\textit{#1}}}}
\newcommand{\ConstantTok}[1]{\textcolor[rgb]{0.00,0.00,0.00}{#1}}
\newcommand{\ControlFlowTok}[1]{\textcolor[rgb]{0.13,0.29,0.53}{\textbf{#1}}}
\newcommand{\DataTypeTok}[1]{\textcolor[rgb]{0.13,0.29,0.53}{#1}}
\newcommand{\DecValTok}[1]{\textcolor[rgb]{0.00,0.00,0.81}{#1}}
\newcommand{\DocumentationTok}[1]{\textcolor[rgb]{0.56,0.35,0.01}{\textbf{\textit{#1}}}}
\newcommand{\ErrorTok}[1]{\textcolor[rgb]{0.64,0.00,0.00}{\textbf{#1}}}
\newcommand{\ExtensionTok}[1]{#1}
\newcommand{\FloatTok}[1]{\textcolor[rgb]{0.00,0.00,0.81}{#1}}
\newcommand{\FunctionTok}[1]{\textcolor[rgb]{0.00,0.00,0.00}{#1}}
\newcommand{\ImportTok}[1]{#1}
\newcommand{\InformationTok}[1]{\textcolor[rgb]{0.56,0.35,0.01}{\textbf{\textit{#1}}}}
\newcommand{\KeywordTok}[1]{\textcolor[rgb]{0.13,0.29,0.53}{\textbf{#1}}}
\newcommand{\NormalTok}[1]{#1}
\newcommand{\OperatorTok}[1]{\textcolor[rgb]{0.81,0.36,0.00}{\textbf{#1}}}
\newcommand{\OtherTok}[1]{\textcolor[rgb]{0.56,0.35,0.01}{#1}}
\newcommand{\PreprocessorTok}[1]{\textcolor[rgb]{0.56,0.35,0.01}{\textit{#1}}}
\newcommand{\RegionMarkerTok}[1]{#1}
\newcommand{\SpecialCharTok}[1]{\textcolor[rgb]{0.00,0.00,0.00}{#1}}
\newcommand{\SpecialStringTok}[1]{\textcolor[rgb]{0.31,0.60,0.02}{#1}}
\newcommand{\StringTok}[1]{\textcolor[rgb]{0.31,0.60,0.02}{#1}}
\newcommand{\VariableTok}[1]{\textcolor[rgb]{0.00,0.00,0.00}{#1}}
\newcommand{\VerbatimStringTok}[1]{\textcolor[rgb]{0.31,0.60,0.02}{#1}}
\newcommand{\WarningTok}[1]{\textcolor[rgb]{0.56,0.35,0.01}{\textbf{\textit{#1}}}}
\usepackage{graphicx,grffile}
\makeatletter
\def\maxwidth{\ifdim\Gin@nat@width>\linewidth\linewidth\else\Gin@nat@width\fi}
\def\maxheight{\ifdim\Gin@nat@height>\textheight\textheight\else\Gin@nat@height\fi}
\makeatother
% Scale images if necessary, so that they will not overflow the page
% margins by default, and it is still possible to overwrite the defaults
% using explicit options in \includegraphics[width, height, ...]{}
\setkeys{Gin}{width=\maxwidth,height=\maxheight,keepaspectratio}
\IfFileExists{parskip.sty}{%
\usepackage{parskip}
}{% else
\setlength{\parindent}{0pt}
\setlength{\parskip}{6pt plus 2pt minus 1pt}
}
\setlength{\emergencystretch}{3em}  % prevent overfull lines
\providecommand{\tightlist}{%
  \setlength{\itemsep}{0pt}\setlength{\parskip}{0pt}}
\setcounter{secnumdepth}{0}
% Redefines (sub)paragraphs to behave more like sections
\ifx\paragraph\undefined\else
\let\oldparagraph\paragraph
\renewcommand{\paragraph}[1]{\oldparagraph{#1}\mbox{}}
\fi
\ifx\subparagraph\undefined\else
\let\oldsubparagraph\subparagraph
\renewcommand{\subparagraph}[1]{\oldsubparagraph{#1}\mbox{}}
\fi

%%% Use protect on footnotes to avoid problems with footnotes in titles
\let\rmarkdownfootnote\footnote%
\def\footnote{\protect\rmarkdownfootnote}

%%% Change title format to be more compact
\usepackage{titling}

% Create subtitle command for use in maketitle
\newcommand{\subtitle}[1]{
  \posttitle{
    \begin{center}\large#1\end{center}
    }
}

\setlength{\droptitle}{-2em}

  \title{R 뿉꽌 썝븯뒗 궎썙뱶쓽 꽕씠踰\textless{}84\textgreater{} 돱뒪瑜
쎒겕濡ㅻ쭅(뒪겕옒븨) 븯뒗 諛⑸쾿}
    \pretitle{\vspace{\droptitle}\centering\huge}
  \posttitle{\par}
    \author{hkim}
    \preauthor{\centering\large\emph}
  \postauthor{\par}
      \predate{\centering\large\emph}
  \postdate{\par}
    \date{2018-11-24}


\begin{document}
\maketitle

\textbf{\emph{preface}} (last update: 2018.11.24) 이번 포스트에서는 R을
이용하여 네이버 뉴스에서 원하는 키워드의 검색 결과를 웹크롤링(스크래핑)
하는 방법에 대하여 설명합니다.

\hypertarget{naver-news-web-scraping-using-keywords-in-r}{%
\section{Naver News Web Scraping using Keywords in
R}\label{naver-news-web-scraping-using-keywords-in-r}}

다음 자료를 참고하였습니다:\\
- \url{https://blog.naver.com/knowch/221060289410}

먼저 필요한 패키지를 설치합니다. 이번 가이드에서 사용하는 패키지는
\texttt{rvest} 와 \texttt{dplyr} 입니다.

\begin{Shaded}
\begin{Highlighting}[]
\CommentTok{## 패키지 설치하기 }
\KeywordTok{install.packages}\NormalTok{(}\StringTok{"rvest"}\NormalTok{)}
\KeywordTok{install.packages}\NormalTok{(}\StringTok{"dplyr"}\NormalTok{)}
\end{Highlighting}
\end{Shaded}

필요한 패키지를 불러옵시다.

\begin{Shaded}
\begin{Highlighting}[]
\CommentTok{## 패키지 불러오기}
\KeywordTok{library}\NormalTok{(rvest)}
\end{Highlighting}
\end{Shaded}

\begin{verbatim}
## Loading required package: xml2
\end{verbatim}

\begin{Shaded}
\begin{Highlighting}[]
\KeywordTok{library}\NormalTok{(dplyr)}
\end{Highlighting}
\end{Shaded}

\begin{verbatim}
## 
## Attaching package: 'dplyr'
\end{verbatim}

\begin{verbatim}
## The following objects are masked from 'package:stats':
## 
##     filter, lag
\end{verbatim}

\begin{verbatim}
## The following objects are masked from 'package:base':
## 
##     intersect, setdiff, setequal, union
\end{verbatim}

\hypertarget{----}{%
\subsection{웹크롤링을 위한 웹주소 리스트 만들기}\label{----}}

웹크롤링 하기 위해서는 웹크롤링의 대상이 되는 웹페이지 주소의 구조를
이해해야 합니다. 네이버 뉴스에서 '미국 기준금리'라는 키워드로 검색하는
경우를 예로 들어 보겠습니다.

먼저 \href{https://news.naver.com/}{네이버 뉴스} 우측 상단에 있는 뉴스
검색에서 '미국+기준금리'를 입력합니다. 날짜도 지정해줍시다.
검색시작일자와 검색종료일자를 각각 2018년 11월 19일로 설정합니다.
뉴스들이 1페이지에서 끝나지 않습니다. 맨 아래에 있는 페이지 숫자를
이리저리 클릭해봅시다. 주소창에서 \texttt{\&start=} 뒤에 있는 숫자가
10단위로 움직이는 것을 확인할 수 있습니다. 다시 1페이지로 돌아오면
다음과 같은 주소를 얻을 수 있습니다.

\begin{verbatim}
https://search.naver.com/search.naver?&where=news&query=미국%2B기준금리&sm=tab_pge&sort=0&photo=0&field=0&reporter_article=&pd=3&ds=2018.11.19&de=2018.11.19&docid=&nso=so:r,p:from20181119to20181119,a:all&mynews=0&cluster_rank=67&start=1&refresh_start=0
\end{verbatim}

위 웹 주소를 하나씩 뜯어보면 아래와 같이 구성되어 있다는 것을 확인할 수
있습니다.

\begin{verbatim}
https://search.naver.com/search.naver?
&where=news
&query=미국%2B기준금리
&sm=tab_pge
&sort=0
&photo=0
&field=0
&reporter_article=
&pd=3
&ds=2018.11.19
&de=2018.11.19
&docid=
&nso=so:r,p:from20181119to20181119,a:all&mynews=0
&cluster_rank=67
&start=1
&refresh_start=0
\end{verbatim}

우리에게 필요한 부분은 \texttt{\&query=미국+기준금리} ,
\texttt{\&ds=2018.11.19} , \texttt{\&de=2018.11.19} , \texttt{\&start=1}
입니다. 이 부분들을 포함해서 검색 결과가 표시되는 웹 주소를 만들어
봅시다. 저는 여러번의 try and error 를 거쳐 아래 주소를 입력하면 우리가
원하는 결과를 얻을 수 있다는 것을 확인했습니다.

\begin{verbatim}
https://search.naver.com/search.naver?&where=news&query=미국+기준금리&pd=3&ds=2018.11.19&de=2018.11.19&start=1
\end{verbatim}

다시 말해, 아래와 같은 주소를 생성하면 원하는 결과를 얻을 수 있습니다.

\begin{verbatim}
https://search.naver.com/search.naver?&where=news&query= [검색키워드] &pd=3&ds= [검색시작날짜] &de= [검색종료날짜] &start= [게시물번호]
\end{verbatim}

날짜·쿼리 등을 원하는대로 바꾸거나, 게시물번호를 11 이나 21 로 바꿔도
원하는 결과를 얻을 수 있습니다.

이러한 구조를 이용하여, 문자열(string)을 결합하여 R 에서 사용할 수 있는
형태로 웹 주소를 만들어 봅시다. 여기에서는 R 에서 문자열을 결합하는
함수로 \texttt{paste0()} 을 사용합니다.

\begin{Shaded}
\begin{Highlighting}[]
\CommentTok{## 변수 입력하기}
\NormalTok{QUERY <-}\StringTok{ "미국+기준금리"} \CommentTok{# 검색키워드}
\NormalTok{DATE  <-}\StringTok{ }\KeywordTok{as.Date}\NormalTok{(}\KeywordTok{as.character}\NormalTok{(}\DecValTok{20181119}\NormalTok{),}\DataTypeTok{format=}\StringTok{"%Y%m%d"}\NormalTok{) }\CommentTok{# 검색시작날짜 & 검색종료날짜}
\NormalTok{DATE <-}\StringTok{ }\KeywordTok{format}\NormalTok{(DATE, }\StringTok{"%Y.%m.%d"}\NormalTok{)}
\NormalTok{PAGE  <-}\StringTok{ }\DecValTok{1}

\NormalTok{naver_url_}\DecValTok{1}\NormalTok{ <-}\StringTok{ "https://search.naver.com/search.naver?&where=news&query="}
\NormalTok{naver_url_}\DecValTok{2}\NormalTok{ <-}\StringTok{ "&pd=3&ds="}
\NormalTok{naver_url_}\DecValTok{3}\NormalTok{ <-}\StringTok{ "&de="}
\NormalTok{naver_url_}\DecValTok{4}\NormalTok{ <-}\StringTok{ "&start="}

\NormalTok{naver_url <-}\StringTok{ }\KeywordTok{paste0}\NormalTok{(naver_url_}\DecValTok{1}\NormalTok{,QUERY,naver_url_}\DecValTok{2}\NormalTok{,DATE,naver_url_}\DecValTok{3}\NormalTok{,DATE,naver_url_}\DecValTok{4}\NormalTok{,PAGE)}
\NormalTok{naver_url}
\end{Highlighting}
\end{Shaded}

\begin{verbatim}
## [1] "https://search.naver.com/search.naver?&where=news&query=미국+기준금리&pd=3&ds=2018.11.19&de=2018.11.19&start=1"
\end{verbatim}

그럼 원하는 주소의 리스트를 만들어 봅시다. 날짜는 11월 19일부터 21일까지
3일간, 게시물은 첫 5 페이지를 살펴보는 것으로 제한하겠습니다.

먼저 날짜 리스트를 만들어 보겠습니다.

\begin{Shaded}
\begin{Highlighting}[]
\CommentTok{## 날짜 리스트 만들기}
\NormalTok{DATE_START <-}\StringTok{ }\KeywordTok{as.Date}\NormalTok{(}\KeywordTok{as.character}\NormalTok{(}\DecValTok{20181119}\NormalTok{), }\DataTypeTok{format=}\StringTok{"%Y%m%d"}\NormalTok{) }\CommentTok{# 시작일자}
\NormalTok{DATE_END   <-}\StringTok{ }\KeywordTok{as.Date}\NormalTok{(}\KeywordTok{as.character}\NormalTok{(}\DecValTok{20181121}\NormalTok{), }\DataTypeTok{format=}\StringTok{"%Y%m%d"}\NormalTok{) }\CommentTok{# 종료일자}
\NormalTok{DATE <-}\StringTok{ }\NormalTok{DATE_START}\OperatorTok{:}\NormalTok{DATE_END}
\NormalTok{DATE <-}\StringTok{ }\KeywordTok{as.Date}\NormalTok{(DATE,}\DataTypeTok{origin=}\StringTok{"1970-01-01"}\NormalTok{)}
\NormalTok{DATE}
\end{Highlighting}
\end{Shaded}

\begin{verbatim}
## [1] "2018-11-19" "2018-11-20" "2018-11-21"
\end{verbatim}

다음은 게시물 번호 리스트를 만들어 보겠습니다. 게시물 번호는 1, 2, 3
으로 증가하는 것이 아니라, 1, 11, 21 로 10씩 증가해야합니다.
\texttt{seq()} 함수를 이용하면 이와 같은 연산을 실행할 수 있습니다.

\begin{Shaded}
\begin{Highlighting}[]
\CommentTok{## 게시물 번호 리스트 만들기}
\NormalTok{PAGE <-}\StringTok{ }\KeywordTok{seq}\NormalTok{(}\DataTypeTok{from=}\DecValTok{1}\NormalTok{,}\DataTypeTok{to=}\DecValTok{41}\NormalTok{,}\DataTypeTok{by=}\DecValTok{10}\NormalTok{) }\CommentTok{# 시작값과 종료값을 지정해줄 수 있습니다.}
\NormalTok{PAGE <-}\StringTok{ }\KeywordTok{seq}\NormalTok{(}\DataTypeTok{from=}\DecValTok{1}\NormalTok{,}\DataTypeTok{by=}\DecValTok{10}\NormalTok{,}\DataTypeTok{length.out=}\DecValTok{5}\NormalTok{) }\CommentTok{# 시작값과 원하는 갯수를 지정할 수도 있습니다.}
\NormalTok{PAGE}
\end{Highlighting}
\end{Shaded}

\begin{verbatim}
## [1]  1 11 21 31 41
\end{verbatim}

이제 \texttt{for} 문을 이용하여 이를 한꺼번에 리스트로 만들겠습니다.

\begin{Shaded}
\begin{Highlighting}[]
\CommentTok{## 네이버 검색결과 url 리스트 만들기}
\NormalTok{naver_url_list <-}\StringTok{ }\KeywordTok{c}\NormalTok{()}
\ControlFlowTok{for}\NormalTok{ (date_i }\ControlFlowTok{in}\NormalTok{ DATE)\{}
  \ControlFlowTok{for}\NormalTok{ (page_i }\ControlFlowTok{in}\NormalTok{ PAGE)\{}
\NormalTok{    dt <-}\StringTok{ }\KeywordTok{format}\NormalTok{(}\KeywordTok{as.Date}\NormalTok{(date_i,}\DataTypeTok{origin=}\StringTok{"1970-01-01"}\NormalTok{), }\StringTok{"%Y.%m.%d"}\NormalTok{)}
\NormalTok{    naver_url <-}\StringTok{ }\KeywordTok{paste0}\NormalTok{(naver_url_}\DecValTok{1}\NormalTok{,QUERY,naver_url_}\DecValTok{2}\NormalTok{,dt,naver_url_}\DecValTok{3}\NormalTok{,dt,naver_url_}\DecValTok{4}\NormalTok{,page_i)}
\NormalTok{    naver_url_list <-}\StringTok{ }\KeywordTok{c}\NormalTok{(naver_url_list, naver_url)}
\NormalTok{  \}}
\NormalTok{\}}
\KeywordTok{head}\NormalTok{(naver_url_list,}\DecValTok{5}\NormalTok{)}
\end{Highlighting}
\end{Shaded}

\begin{verbatim}
## [1] "https://search.naver.com/search.naver?&where=news&query=미국+기준금리&pd=3&ds=2018.11.19&de=2018.11.19&start=1" 
## [2] "https://search.naver.com/search.naver?&where=news&query=미국+기준금리&pd=3&ds=2018.11.19&de=2018.11.19&start=11"
## [3] "https://search.naver.com/search.naver?&where=news&query=미국+기준금리&pd=3&ds=2018.11.19&de=2018.11.19&start=21"
## [4] "https://search.naver.com/search.naver?&where=news&query=미국+기준금리&pd=3&ds=2018.11.19&de=2018.11.19&start=31"
## [5] "https://search.naver.com/search.naver?&where=news&query=미국+기준금리&pd=3&ds=2018.11.19&de=2018.11.19&start=41"
\end{verbatim}

우리가 웹크롤링하고자 하는 주소 리스트가 생성되었습니다.

\hypertarget{----}{%
\subsection{검색결과에 포함된 링크 목록 만들기}\label{----}}

이제 본격적으로 웹 크롤링을 실시하겠습니다. 우선 위에서 만든 주소 리스트
가운데 하나를 가져와 분석합시다.

\begin{verbatim}
https://search.naver.com/search.naver?&where=news&query=미국+기준금리&pd=3&ds=2018.11.19&de=2018.11.19&start=1
\end{verbatim}

우리에게 필요한 것은 이 페이지에서 나타난 네이버뉴스 링크입니다. 일반
신문사 링크를 가져오면 제각각 홈페이지 구조가 달라서 웹크롤링이
힘들어집니다.

먼저 페이지의 소스를 뜯어봅시다. 웹브라우저로 크롬을 쓰는 것을
추천합니다. F12 를 누르면 페이지 소스를 뜯어볼 수 있습니다.

(스크린샷을 동반한 상세한 설명이 필요한 부분이지만 일단 넘어가겠습니다.
필요시 참고 웹사이트를 찾아보세요.)

우리가 원하는 네이버 뉴스 링크들은 \texttt{id="main\_pack"} ,
\texttt{class="news\ ..."} , \texttt{class="type01"} 아래에 있으며
\texttt{href=} 라는 attribute 에 있는 것을 확인할 수 있습니다. html
소스를 분석하여 이러한 조건을 만족하는 값들만 추려내면 아래와 같습니다.

\begin{Shaded}
\begin{Highlighting}[]
\NormalTok{naver_url <-}\StringTok{ "https://search.naver.com/search.naver?&where=news&query=미국+기준금리&pd=3&ds=2018.11.19&de=2018.11.19&start=1"}
\NormalTok{html <-}\StringTok{ }\KeywordTok{read_html}\NormalTok{(naver_url)}
\NormalTok{temp <-}\StringTok{ }\KeywordTok{unique}\NormalTok{(}\KeywordTok{html_nodes}\NormalTok{(html,}\StringTok{'#main_pack'}\NormalTok{)}\OperatorTok\StringTok{ }\CommentTok{# id= 는 # 을 붙인다}
\StringTok{                 }\KeywordTok{html_nodes}\NormalTok{(}\DataTypeTok{css=}\StringTok{'.news '}\NormalTok{)}\OperatorTok\StringTok{    }\CommentTok{# class= 는 css= 를 붙인다 }
\StringTok{                 }\KeywordTok{html_nodes}\NormalTok{(}\DataTypeTok{css=}\StringTok{'.type01'}\NormalTok{)}\OperatorTok
\StringTok{                 }\KeywordTok{html_nodes}\NormalTok{(}\StringTok{'a'}\NormalTok{)}\OperatorTok
\StringTok{                 }\KeywordTok{html_attr}\NormalTok{(}\StringTok{'href'}\NormalTok{))}

\KeywordTok{head}\NormalTok{(temp,}\DecValTok{5}\NormalTok{)}
\end{Highlighting}
\end{Shaded}

\begin{verbatim}
## [1] "http://news.einfomax.co.kr/news/articleView.html?idxno=4003244"                       
## [2] "#"                                                                                    
## [3] "http://www.seoulwire.com/news/articleView.html?idxno=35031"                           
## [4] "http://news.mt.co.kr/mtview.php?no=2018111911542399654"                               
## [5] "https://news.naver.com/main/read.nhn?mode=LSD&mid=sec&sid1=004&oid=008&aid=0004134391"
\end{verbatim}

\texttt{news.naver.com} 으로 시작하는 웹사이트 외에도 많은 사이트들이
들어가있는 것을 확인할 수 있습니다. 이것들을 지우는 것은 나중에 한꺼번에
처리하겠습니다.

\begin{Shaded}
\begin{Highlighting}[]
\CommentTok{## 네이버 검색결과 url 리스트에서 관련기사 url 리스트 만들기}
\NormalTok{news_url <-}\StringTok{ }\KeywordTok{c}\NormalTok{()}
\NormalTok{news_date <-}\KeywordTok{c}\NormalTok{() }

\ControlFlowTok{for}\NormalTok{ (date_i }\ControlFlowTok{in}\NormalTok{ DATE)\{}
  \ControlFlowTok{for}\NormalTok{ (page_i }\ControlFlowTok{in}\NormalTok{ PAGE)\{}
\NormalTok{    dt <-}\StringTok{ }\KeywordTok{format}\NormalTok{(}\KeywordTok{as.Date}\NormalTok{(date_i,}\DataTypeTok{origin=}\StringTok{"1970-01-01"}\NormalTok{), }\StringTok{"%Y.%m.%d"}\NormalTok{)}
\NormalTok{    naver_url <-}\StringTok{ }\KeywordTok{paste0}\NormalTok{(naver_url_}\DecValTok{1}\NormalTok{,QUERY,naver_url_}\DecValTok{2}\NormalTok{,dt,naver_url_}\DecValTok{3}\NormalTok{,dt,naver_url_}\DecValTok{4}\NormalTok{,page_i)}
\NormalTok{    html <-}\StringTok{ }\KeywordTok{read_html}\NormalTok{(naver_url)}
\NormalTok{    temp <-}\StringTok{ }\KeywordTok{unique}\NormalTok{(}\KeywordTok{html_nodes}\NormalTok{(html,}\StringTok{'#main_pack'}\NormalTok{)}\OperatorTok\StringTok{ }\CommentTok{# id= 는 # 을 붙인다}
\StringTok{                     }\KeywordTok{html_nodes}\NormalTok{(}\DataTypeTok{css=}\StringTok{'.news '}\NormalTok{)}\OperatorTok\StringTok{    }\CommentTok{# class= 는 css= 를 붙인다 }
\StringTok{                     }\KeywordTok{html_nodes}\NormalTok{(}\DataTypeTok{css=}\StringTok{'.type01'}\NormalTok{)}\OperatorTok
\StringTok{                     }\KeywordTok{html_nodes}\NormalTok{(}\StringTok{'a'}\NormalTok{)}\OperatorTok
\StringTok{                     }\KeywordTok{html_attr}\NormalTok{(}\StringTok{'href'}\NormalTok{))}
\NormalTok{    news_url <-}\StringTok{ }\KeywordTok{c}\NormalTok{(news_url,temp)}
\NormalTok{    news_date <-}\StringTok{ }\KeywordTok{c}\NormalTok{(news_date,}\KeywordTok{rep}\NormalTok{(dt,}\KeywordTok{length}\NormalTok{(temp)))}
\NormalTok{  \}}
  \KeywordTok{print}\NormalTok{(dt) }\CommentTok{# 진행상황을 알기 위함이니 속도가 느려지면 제외}
\NormalTok{\}}

\NormalTok{NEWS0 <-}\StringTok{ }\KeywordTok{as.data.frame}\NormalTok{(}\KeywordTok{cbind}\NormalTok{(}\DataTypeTok{date=}\NormalTok{news_date, }\DataTypeTok{url=}\NormalTok{news_url, }\DataTypeTok{query=}\NormalTok{QUERY))}
\NormalTok{NEWS1 <-}\StringTok{ }\NormalTok{NEWS0[}\KeywordTok{which}\NormalTok{(}\KeywordTok{grepl}\NormalTok{(}\StringTok{"news.naver.com"}\NormalTok{,NEWS0}\OperatorTok{$}\NormalTok{url)),]         }\CommentTok{# 네이버뉴스(news.naver.com)만 대상으로 한다   }
\NormalTok{NEWS1 <-}\StringTok{ }\NormalTok{NEWS1[}\KeywordTok{which}\NormalTok{(}\OperatorTok{!}\KeywordTok{grepl}\NormalTok{(}\StringTok{"sports.news.naver.com"}\NormalTok{,NEWS1}\OperatorTok{$}\NormalTok{url)),] }\CommentTok{# 스포츠뉴스(sports.news.naver.com)는 제외한다  }
\NormalTok{NEWS2 <-}\StringTok{ }\NormalTok{NEWS1[}\OperatorTok{!}\KeywordTok{duplicated}\NormalTok{(NEWS1), ] }\CommentTok{# 중복된 링크 제거 }
\end{Highlighting}
\end{Shaded}

첫번째 url
\url{https://news.naver.com/main/read.nhn?mode=LSD\&mid=sec\&sid1=004\&oid=008\&aid=0004134391}
을 웹브라우저 주소창에 입력해봅시다. 원하는 뉴스가 정상적으로 출력되는
것을 확인할 수 있습니다.

\hypertarget{----}{%
\subsection{뉴스 기사의 제목과 본문 추출하기}\label{----}}

거의 다 왔습니다. 지금까지 우리는 원하는 기간 동안 원하는 키워드를
포함한 뉴스의 url 주소를 수집하는데 성공했습니다. 그럼 이 url 주소를
이용하여 기사의 제목과 본문을 추출해 봅시다.

첫번째 url
\url{https://news.naver.com/main/read.nhn?mode=LSD\&mid=sec\&sid1=004\&oid=008\&aid=0004134391}
을 웹브라우저 주소창에 입력해봅시다. 그리고 마찬가지로 F12 를 눌러
소스를 뜯어보겠습니다.

분석 결과, 기사의 제목은 \texttt{id="articleTitle"} 아래에 있고 본문은
\texttt{id="articleBodyContents"} 아래에 있는 것을 확인했습니다. 앞에서
생성한 url 리스트를 이용하여 각 기사의 제목과 본문을 추출하고, 중복된
문자열을 제거한 최종 결과를 \texttt{NEWS} 에 저장하겠습니다.

\begin{Shaded}
\begin{Highlighting}[]
\CommentTok{## 뉴스 페이지에 있는 기사의 제목과 본문을 크롤링}
\NormalTok{NEWS2}\OperatorTok{$}\NormalTok{news_title   <-}\StringTok{ ""}
\NormalTok{NEWS2}\OperatorTok{$}\NormalTok{news_content <-}\StringTok{ ""}

\ControlFlowTok{for}\NormalTok{ (i }\ControlFlowTok{in} \DecValTok{1}\OperatorTok{:}\KeywordTok{dim}\NormalTok{(NEWS2)[}\DecValTok{1}\NormalTok{])\{}
\NormalTok{  html <-}\StringTok{ }\KeywordTok{read_html}\NormalTok{(}\KeywordTok{as.character}\NormalTok{(NEWS2}\OperatorTok{$}\NormalTok{url[i]))}
\NormalTok{  temp_news_title   <-}\StringTok{ }\KeywordTok{repair_encoding}\NormalTok{(}\KeywordTok{html_text}\NormalTok{(}\KeywordTok{html_nodes}\NormalTok{(html,}\StringTok{'#articleTitle'}\NormalTok{)),}\DataTypeTok{from =} \StringTok{'utf-8'}\NormalTok{)}
\NormalTok{  temp_news_content <-}\StringTok{ }\KeywordTok{repair_encoding}\NormalTok{(}\KeywordTok{html_text}\NormalTok{(}\KeywordTok{html_nodes}\NormalTok{(html,}\StringTok{'#articleBodyContents'}\NormalTok{)),}\DataTypeTok{from =} \StringTok{'utf-8'}\NormalTok{)}
  \ControlFlowTok{if}\NormalTok{ (}\KeywordTok{length}\NormalTok{(temp_news_title)}\OperatorTok{>}\DecValTok{0}\NormalTok{)\{}
\NormalTok{    NEWS2}\OperatorTok{$}\NormalTok{news_title[i]   <-}\StringTok{ }\NormalTok{temp_news_title}
\NormalTok{    NEWS2}\OperatorTok{$}\NormalTok{news_content[i] <-}\StringTok{ }\NormalTok{temp_news_content}
\NormalTok{  \}}
\NormalTok{\}}

\NormalTok{NEWS2}\OperatorTok{$}\NormalTok{news_content <-}\StringTok{ }\KeywordTok{gsub}\NormalTok{(}\StringTok{"// flash 오류를 우회하기 위한 함수 추가}\CharTok{\textbackslash{}n}\StringTok{function _flash_removeCallback()"}\NormalTok{, }\StringTok{""}\NormalTok{, NEWS2}\OperatorTok{$}\NormalTok{news_content) }\CommentTok{# 중복된 문자열 제거}
\NormalTok{NEWS <-}\StringTok{ }\NormalTok{NEWS2 }\CommentTok{# 최종 결과 저장}
\KeywordTok{save}\NormalTok{(NEWS, }\DataTypeTok{file=}\StringTok{"./DATA0/NEWS.RData"}\NormalTok{) }\CommentTok{# working directory 아래 subfolder "DATA0" 에 저장}
\end{Highlighting}
\end{Shaded}

\hypertarget{-}{%
\subsection{전체 코드}\label{-}}

이번 포스트에서 다룬 R을 이용하여 네이버 뉴스에서 원하는 키워드의 검색
결과를 웹크롤링(스크래핑) 하기 위한 전체 코드는 아래와 같습니다.

\begin{Shaded}
\begin{Highlighting}[]
\CommentTok{## 네이버 뉴스에서 원하는 키워드의 검색 결과를 웹크롤링(스크래핑)하는 코드}
\CommentTok{## 제작: hkim (dr-hkim.github.io)}

\CommentTok{## 패키지 불러오기}
\KeywordTok{library}\NormalTok{(rvest)}
\KeywordTok{library}\NormalTok{(dplyr)}

\CommentTok{## 변수 입력하기}
\NormalTok{QUERY <-}\StringTok{ "미국+기준금리"} \CommentTok{# 검색키워드}
\NormalTok{DATE  <-}\StringTok{ }\KeywordTok{as.Date}\NormalTok{(}\KeywordTok{as.character}\NormalTok{(}\DecValTok{20181119}\NormalTok{),}\DataTypeTok{format=}\StringTok{"%Y%m%d"}\NormalTok{) }\CommentTok{# 검색시작날짜 & 검색종료날짜}
\NormalTok{DATE <-}\StringTok{ }\KeywordTok{format}\NormalTok{(DATE, }\StringTok{"%Y.%m.%d"}\NormalTok{)}
\NormalTok{PAGE  <-}\StringTok{ }\DecValTok{1}

\NormalTok{naver_url_}\DecValTok{1}\NormalTok{ <-}\StringTok{ "https://search.naver.com/search.naver?&where=news&query="}
\NormalTok{naver_url_}\DecValTok{2}\NormalTok{ <-}\StringTok{ "&pd=3&ds="}
\NormalTok{naver_url_}\DecValTok{3}\NormalTok{ <-}\StringTok{ "&de="}
\NormalTok{naver_url_}\DecValTok{4}\NormalTok{ <-}\StringTok{ "&start="}

\CommentTok{## 날짜 리스트 만들기}
\NormalTok{DATE_START <-}\StringTok{ }\KeywordTok{as.Date}\NormalTok{(}\KeywordTok{as.character}\NormalTok{(}\DecValTok{20181119}\NormalTok{),}\DataTypeTok{format=}\StringTok{"%Y%m%d"}\NormalTok{) }\CommentTok{# 시작일자}
\NormalTok{DATE_END   <-}\StringTok{ }\KeywordTok{as.Date}\NormalTok{(}\KeywordTok{as.character}\NormalTok{(}\DecValTok{20181121}\NormalTok{),  }\DataTypeTok{format=}\StringTok{"%Y%m%d"}\NormalTok{) }\CommentTok{# 종료일자}
\NormalTok{DATE <-}\StringTok{ }\NormalTok{DATE_START}\OperatorTok{:}\NormalTok{DATE_END}
\NormalTok{DATE <-}\StringTok{ }\KeywordTok{as.Date}\NormalTok{(DATE,}\DataTypeTok{origin=}\StringTok{"1970-01-01"}\NormalTok{)}

\CommentTok{## 게시물 번호 리스트 만들기}
\NormalTok{PAGE <-}\StringTok{ }\KeywordTok{seq}\NormalTok{(}\DataTypeTok{from=}\DecValTok{1}\NormalTok{,}\DataTypeTok{to=}\DecValTok{41}\NormalTok{,}\DataTypeTok{by=}\DecValTok{10}\NormalTok{) }\CommentTok{# 시작값과 종료값을 지정해줄 수 있습니다.}
\NormalTok{PAGE <-}\StringTok{ }\KeywordTok{seq}\NormalTok{(}\DataTypeTok{from=}\DecValTok{1}\NormalTok{,}\DataTypeTok{by=}\DecValTok{10}\NormalTok{,}\DataTypeTok{length.out=}\DecValTok{5}\NormalTok{) }\CommentTok{# 시작값과 원하는 갯수를 지정할 수도 있습니다.}

\CommentTok{## 네이버 검색결과 url 리스트에서 관련기사 url 리스트 만들기}
\NormalTok{news_url <-}\StringTok{ }\KeywordTok{c}\NormalTok{()}
\NormalTok{news_date <-}\KeywordTok{c}\NormalTok{() }

\ControlFlowTok{for}\NormalTok{ (date_i }\ControlFlowTok{in}\NormalTok{ DATE)\{}
  \ControlFlowTok{for}\NormalTok{ (page_i }\ControlFlowTok{in}\NormalTok{ PAGE)\{}
\NormalTok{    dt <-}\StringTok{ }\KeywordTok{format}\NormalTok{(}\KeywordTok{as.Date}\NormalTok{(date_i,}\DataTypeTok{origin=}\StringTok{"1970-01-01"}\NormalTok{), }\StringTok{"%Y.%m.%d"}\NormalTok{)}
\NormalTok{    naver_url <-}\StringTok{ }\KeywordTok{paste0}\NormalTok{(naver_url_}\DecValTok{1}\NormalTok{,QUERY,naver_url_}\DecValTok{2}\NormalTok{,dt,naver_url_}\DecValTok{3}\NormalTok{,dt,naver_url_}\DecValTok{4}\NormalTok{,page_i)}
\NormalTok{    html <-}\StringTok{ }\KeywordTok{read_html}\NormalTok{(naver_url)}
\NormalTok{    temp <-}\StringTok{ }\KeywordTok{unique}\NormalTok{(}\KeywordTok{html_nodes}\NormalTok{(html,}\StringTok{'#main_pack'}\NormalTok{)}\OperatorTok\StringTok{ }\CommentTok{# id= 는 # 을 붙인다}
\StringTok{                     }\KeywordTok{html_nodes}\NormalTok{(}\DataTypeTok{css=}\StringTok{'.news '}\NormalTok{)}\OperatorTok\StringTok{    }\CommentTok{# class= 는 css= 를 붙인다 }
\StringTok{                     }\KeywordTok{html_nodes}\NormalTok{(}\DataTypeTok{css=}\StringTok{'.type01'}\NormalTok{)}\OperatorTok
\StringTok{                     }\KeywordTok{html_nodes}\NormalTok{(}\StringTok{'a'}\NormalTok{)}\OperatorTok
\StringTok{                     }\KeywordTok{html_attr}\NormalTok{(}\StringTok{'href'}\NormalTok{))}
\NormalTok{    news_url <-}\StringTok{ }\KeywordTok{c}\NormalTok{(news_url,temp)}
\NormalTok{    news_date <-}\StringTok{ }\KeywordTok{c}\NormalTok{(news_date,}\KeywordTok{rep}\NormalTok{(dt,}\KeywordTok{length}\NormalTok{(temp)))}
\NormalTok{  \}}
  \KeywordTok{print}\NormalTok{(dt) }\CommentTok{# 진행상황을 알기 위함이니 속도가 느려지면 제외}
\NormalTok{\}}

\NormalTok{NEWS0 <-}\StringTok{ }\KeywordTok{as.data.frame}\NormalTok{(}\KeywordTok{cbind}\NormalTok{(}\DataTypeTok{date=}\NormalTok{news_date, }\DataTypeTok{url=}\NormalTok{news_url, }\DataTypeTok{query=}\NormalTok{QUERY))}
\NormalTok{NEWS1 <-}\StringTok{ }\NormalTok{NEWS0[}\KeywordTok{which}\NormalTok{(}\KeywordTok{grepl}\NormalTok{(}\StringTok{"news.naver.com"}\NormalTok{,NEWS0}\OperatorTok{$}\NormalTok{url)),]         }\CommentTok{# 네이버뉴스(news.naver.com)만 대상으로 한다   }
\NormalTok{NEWS1 <-}\StringTok{ }\NormalTok{NEWS1[}\KeywordTok{which}\NormalTok{(}\OperatorTok{!}\KeywordTok{grepl}\NormalTok{(}\StringTok{"sports.news.naver.com"}\NormalTok{,NEWS1}\OperatorTok{$}\NormalTok{url)),] }\CommentTok{# 스포츠뉴스(sports.news.naver.com)는 제외한다  }
\NormalTok{NEWS2 <-}\StringTok{ }\NormalTok{NEWS1[}\OperatorTok{!}\KeywordTok{duplicated}\NormalTok{(NEWS1), ] }\CommentTok{# 중복된 링크 제거 }


\CommentTok{## 뉴스 페이지에 있는 기사의 제목과 본문을 크롤링}
\NormalTok{NEWS2}\OperatorTok{$}\NormalTok{news_title   <-}\StringTok{ ""}
\NormalTok{NEWS2}\OperatorTok{$}\NormalTok{news_content <-}\StringTok{ ""}

\ControlFlowTok{for}\NormalTok{ (i }\ControlFlowTok{in} \DecValTok{1}\OperatorTok{:}\KeywordTok{dim}\NormalTok{(NEWS2)[}\DecValTok{1}\NormalTok{])\{}
\NormalTok{  html <-}\StringTok{ }\KeywordTok{read_html}\NormalTok{(}\KeywordTok{as.character}\NormalTok{(NEWS2}\OperatorTok{$}\NormalTok{url[i]))}
\NormalTok{  temp_news_title   <-}\StringTok{ }\KeywordTok{repair_encoding}\NormalTok{(}\KeywordTok{html_text}\NormalTok{(}\KeywordTok{html_nodes}\NormalTok{(html,}\StringTok{'#articleTitle'}\NormalTok{)),}\DataTypeTok{from =} \StringTok{'utf-8'}\NormalTok{)}
\NormalTok{  temp_news_content <-}\StringTok{ }\KeywordTok{repair_encoding}\NormalTok{(}\KeywordTok{html_text}\NormalTok{(}\KeywordTok{html_nodes}\NormalTok{(html,}\StringTok{'#articleBodyContents'}\NormalTok{)),}\DataTypeTok{from =} \StringTok{'utf-8'}\NormalTok{)}
  \ControlFlowTok{if}\NormalTok{ (}\KeywordTok{length}\NormalTok{(temp_news_title)}\OperatorTok{>}\DecValTok{0}\NormalTok{)\{}
\NormalTok{    NEWS2}\OperatorTok{$}\NormalTok{news_title[i]   <-}\StringTok{ }\NormalTok{temp_news_title}
\NormalTok{    NEWS2}\OperatorTok{$}\NormalTok{news_content[i] <-}\StringTok{ }\NormalTok{temp_news_content}
\NormalTok{  \}}
\NormalTok{\}}

\NormalTok{NEWS2}\OperatorTok{$}\NormalTok{news_content <-}\StringTok{ }\KeywordTok{gsub}\NormalTok{(}\StringTok{"// flash 오류를 우회하기 위한 함수 추가}\CharTok{\textbackslash{}n}\StringTok{function _flash_removeCallback()"}\NormalTok{, }\StringTok{""}\NormalTok{, NEWS2}\OperatorTok{$}\NormalTok{news_content)}
\NormalTok{NEWS <-}\StringTok{ }\NormalTok{NEWS2 }\CommentTok{# 최종 결과 저장}
\KeywordTok{save}\NormalTok{(NEWS, }\DataTypeTok{file=}\StringTok{"./DATA0/NEWS.RData"}\NormalTok{) }\CommentTok{# working directory 아래 subfolder "DATA0" 에 저장}
\end{Highlighting}
\end{Shaded}


\end{document}
